% !TEX root = CSC104LectureNotes.tex

\setcounter{chapter}{7}
\chapter{Debugging Conditionals}

\topquote{A good programmer is someone who always looks both ways before crossing a one-way street.}{Doug Linder}

\minitoc

\section{Debugging Conditionals}

Virtually every program you ever write will, at some point in its development, be wrong.  Learning how to find and fix mistakes in code is at least as important a skill to develop as learning how to write code in the first place.  The exercises you will work on in class today will help you discover what these skills are, and how much work you have ahead of you in developing them.

In general, a good debugger is someone who can accomplish the following:

\begin{itemize}
    \item \textbf{Understand what the program should do.}  It seems obvious, but sometimes we can get so mired down in one little detail of a program that we can forget what it is we're trying to accomplish.  Identifying a bug begins with \textit{articulating the correct behavior}.  If you can't tell someone what the program is supposed to do, do you really know whether it's doing it?
    \item \textbf{Understand what the program says it will do.}  It is easy for to us read something -- a computer program, a term paper, a blog post -- and assume it says something after reading very little of it.  Ask your psychology professor how our brains perform \textit{chunking}, and how this can lead to information processing errors.  You may not have even noticed the typo in the first sentence of this bulleted item.  Don't read what you think is there; read what's there.
    \item \textbf{Be humble.}  Don't assume that it works just because you wrote it.
    \item \textbf{Rest often.}  Every problem becomes temporarily insurmountable at some point.  If you can't fix an error, let your brain, your eyes, and your body relax for a little bit, and then come back to the problem refreshed.  Resist the temptation to keep pushing through when you have nothing left, or to assume that taking breaks is for the weak.  Most bugs get found and fixed by fresh heads.
\end{itemize}