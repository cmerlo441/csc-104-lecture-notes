% !TEX root = CSC104LectureNotes.tex

\chapter{Validating User Input}

\topquote{Something.}{Someone}

\minitoc

On Day \ref{day:interactive-graphics} we wrote a program that asked for user input, and quit if the input was invalid.  You are probably aware that real-world programs don't -- or at least shouldn't -- respond this way.  It would be much more user-friendly if our program could give the user another chance instead of just exiting.  Today we will learn a good technique for making this happen.

\section{Introduction to Loops}

In Section \ref{sec:threekinds} of these notes, you learned that there are three fundamental kinds of computer statements: sequential statements, selection statements, and iteration statements.  It is now time to learn about the last of these.

\section{Iteration}

One of the first goals of computer designers, even going back to Sir Charles Babbage,\footnote{\url{https://www.computerhistory.org/babbage/charlesbabbage/}} was that they would make life simpler by doing the sort of tasks that we humans didn't want to do.  In Chapter 4 of zyBooks, you will learn how Python makes it easy for us to tell the computer to automate certain tasks.

\section{Syntax}

Take care to notice that the syntax of a \addindex{Statement!Loop}{loop statement} is similar to that of a conditional statement: there's a keyword, a condition, a colon, and then a bunch of indented lines.  See these examples:

\begin{py}{If Statements and While Statements}{if-and-while}
x = 1

if x < 5:
    print("Hello from the if statement!")
    x += 1

print("More stuff happens here.")

while x < 5:
    print("Hello from the while statement!")
    x += 1

print("More stuff happens here.")
\end{py}

There's a big difference between the \mintinline{py}{if} statement and the \mintinline{py}{while} statement, however.  The body of the \mintinline{py}{if} statement runs at most once, but the body of the \mintinline{py}{while} loop will keep repeating as long as the condition is true.

Before class, see if you can determine what the outcome of this program will be, and then type it in and run it to see if you were right.
