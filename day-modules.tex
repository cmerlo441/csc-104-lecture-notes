% !TEX root = CSC104LectureNotes.tex

\chapter{Modules}
\label{day:modules}

\topquote{A computer is like a violin.  You can imagine a novice trying first a phonograph and then a violin. The latter, he says, sounds terrible.  That is the argument we have heard from our humanists and most of our computer scientists.  Computer programs are good, they say, for particular purposes, but they aren’t flexible.  Neither is a violin, or a typewriter, until you learn how to use it.}{Marvin Minsky}

\minitoc

\section{Where Your Files Go When You Save Them}

\subsection{The Department's Computers}

When you log in to the Windows computers in the Department of Mathematics, Computer Science, and Information Technology's labs in B Cluster, these computers make a couple of interesting resources available to you.  Of the most interest right now is a hard drive that is connected over the Department's network and mapped to drive \texttt{J:\textbackslash} (or ``the J drive'').  Because this is a network-mapped drive, you will always attach to the same J drive when you log in to any one of the computers in our computerized classrooms, or the Computer Learning Center in B 225, or even an instructor's computer in his or her faculty office.

For this reason, it is \textbf{\textit{strongly suggested}} that you save all files you create on one of the Department's computers to the J drive -- this way, there's less to remember.

It is important to know that any file you save to the C drive (the locally-mounted hard drive) on the Department's Windows computers is \textit{very likely to not exist} when you go looking for it.

\subsection{Your Computer}

Do you use your own computer during class?  Do you, or will you, use your own computer at home?  If so, you may already have an organization system set up, and you don't have to think much about where your new Python programs are going to be saved.  If not, then your computer is likely to suggest a suitable location.

In any case, make sure you know where your computer is storing files.  If you aren't sure of the location, it will be incredibly difficult for someone else to help you find these files later.

\section{Three Kinds of Python Programs}

As you can see in Section 2.8 of zyBooks, there are three basic ways of using IDLE to create Python programs.  So far, we have only been using the first, called \textit{interactive mode}.  Today, you will learn how to create \addindex{Script}{scripts} and \addindex{Module}{modules}.  Be sure to read Sections 2.8 and 2.9 carefully before class, pay attention during lecture, and ask any questions you have to as soon as they come up.  The work you do today will help you to understand these concepts, and create the tools you have to create when necessary.