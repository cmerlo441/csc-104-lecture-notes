% !TEX root = CSC104LectureNotes.tex

\chapter{The Software Development Life Cycle}
\label{day:bagels}

\topquote{Progress is possible only if we train ourselves to think about programs without thinking of them as pieces of executable code.}{Edsger W.\ Dijkstra}

\section{The Software Development Life Cycle}

The task of programming starts away from the computer.  Programmers don't just sit at a computer and start writing code that works; it takes careful planning and consideration to create a program that not only performs the job the programmer intended, but also addresses the need that first caused the programmer to start planning and writing.

Many thoughtful writers have tried to explain the steps involved in successful software creation, and so there is no one standard way to express what we do and how we do it.  But most expressions of the \addindex{Software Development Life Cycle}{software development life cycle} center around these six stages:

\begin{enumerate}
    \item \textbf{Understand the problem and the intended outcome.}  More than one programmer has developed a wonderful program for a client that doesn't do any of what the client asked for.
    \item \textbf{Design the solution.}  Once you understand what has to be done, fit the specification into your mode of working.  In this case, take the project requirements and turn them into an outline of a Python program.
    \item \textbf{Write the program.}  Once you have an idea of what you're going to write, sit down and write it.
    \item \textbf{Test and debug the program.}  The likelihood that your program is going to work the way you want the first time is inversely proportional to how important the program is.  Make sure it works, and fix it when it doesn't.
    \item \textbf{Create documentation.}  We have to write headers and comments, and sometimes we have to write blog posts and wiki articles and product manuals.
    \item \textbf{Deploy and maintain the product.}  For many programmers, the job isn't done when the program is written.  Some of us are responsible for making sure a program continues to work for months or even years after it first went into production.
\end{enumerate}

Today in class you are going to have some opportunities to work through the software development life cycle as we create some programs from problem description all the way through to finished product.