% !TEX root = CSC104LectureNotes.tex

\chapter{Tracing a Python Program}
\label{day:tracing-python-1}

\topquote{Debugging is twice as hard as writing the code in the first place.  Therefore, if you write the code as cleverly as possible, you are, by definition, not smart enough to debug it.}{Brian W.\ Kernighan}

\section{Tracing a Python Program}

Successful programmers are adept at determining what a program will do just by reading the source code.  This is true for several reasons, but one critical one is that it helps us find our mistakes.  We tend to make assumptions about how a program will run, and when these assumptions are not borne out in the program's results, it is our job to find out why.  In short, we must figure out the difference between \textit{what we think} the program should do, and \textit{what the program says} it will do. 

Today's tasks will help you develop this skill.  Specifically, you will learn how to determine what happens when the user types input into our program, how and when the data inside a variable changes, and how to describe what will appear on the screen when a program runs.

Take today's activities seriously -- especially if you plan to take more Computer Science or Information Technology courses after this one -- because you will rely on these skills over and over again.